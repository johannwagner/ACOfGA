\documentclass[conference]{IEEEtran}
\IEEEoverridecommandlockouts
% The preceding line is only needed to identify funding in the first footnote. If that is unneeded, please comment it out.
\usepackage{cite}
\usepackage{amsmath,amssymb,amsfonts}
\usepackage{algorithmic}
\usepackage{graphicx}
\usepackage{textcomp}
\def\BibTeX{{\rm B\kern-.05em{\sc i\kern-.025em b}\kern-.08em
    T\kern-.1667em\lower.7ex\hbox{E}\kern-.125emX}}
\begin{document}

\title{Automatic Code Optimizations on GPU Architectures}

\author{
	\IEEEauthorblockN{Johann Wagner}
	\IEEEauthorblockA{
		\textit{Otto-von-Guericke Universität} \\
		Magdeburg, Germany\\
		johann.wagner@st.ovgu.de
	}
}


\maketitle

\begin{abstract}

\end{abstract}

\begin{IEEEkeywords}
GPU Compiler Optimizations
\end{IEEEkeywords}

\section{Introduction}

	% Limit Topic
		% Limited to CUDA 
			% Partition Camping is an effect from Nvidia			
		% Only Compiler Optimization 
		% No Optimization on Run Time
	

\section{Background}

	\subsection{General Purpose GPUs}
	
		% Explanation, why GPGPU is interessting
			% Multithreading
		% Advantages, Disadvantages of GPGPU
		
	\subsection{NVIDIA CUDA}
		
		% Explanation
		% Information about CUDA Modell
	
	\subsection{Memory Seperation in CUDA}
		
		% Explanation about the six different memory types.
		% Categorization with Size, Availability, Speed, Access Time 

\section{Body}

	% Introduction
	% Why do we want to optimize Memory Access
	% Why do we want to do it automagically
	
	% Each Optimization should get at least two sentences, which should outline those categories.
	% Categories
		% Use Cases (special or general)
		% Efficiency
		% Difficulty
		% Error Case
	
	\subsection{Thread-Block Merging}
		
		% General Explanation 
		% Graphic for Explanation
	
	\subsection{Thread Merging}
		
		% General Explanation 
		% Difference to Thread Block Merging
		% Graphic for Explanation
		
	\subsection{Data Prefetching}
		
		% General Explanation
		
	\subsection{Partition Camping}
		
		% General Explanation
		% Graphics
			% Not that easy to understand

\section{Evaluation}
	
	% No Idea

\section{Discussion}

	% Go through Categories and categorize every Optimization for every Category and explain why.
	% Categories
		% Use Cases (special or general)
		% Efficiency
		% Difficulty
			% Code Examples (?)
		% Error Case


\section{Related Work}

\section{Conclusion}

\begin{thebibliography}{00}
\bibitem{b1} G. Eason, B. Noble, and I. N. Sneddon, ``On certain integrals of Lipschitz-Hankel type involving products of Bessel functions,'' Phil. Trans. Roy. Soc. London, vol. A247, pp. 529--551, April 1955.

\end{thebibliography}

\end{document}

